%%% The weekly report template of DTao's lab
%%% Designed by Di Hu
%%% Implemented by Yake Wei
%%% 2020-9-23


\documentclass[UTF8,10.5pt]{article} % 10.5pt 为字号大小
\usepackage{amssymb,amsfonts,amsmath,amsthm}
\usepackage{fancyhdr}
\usepackage{xeCJK}
\usepackage{enumitem}
\usepackage[numbers,sort&compress]{natbib}\setlength{\bibsep}{0.5ex}
\usepackage{amsmath}
\usepackage{indentfirst}
\usepackage{fontspec}
\usepackage[a4paper]{geometry}
\usepackage{graphicx}
\usepackage{color}
\usepackage{fancyhdr}
\setenumerate[1]{itemsep=0pt,partopsep=0pt,parsep=\parskip,topsep=5pt}
\setCJKmainfont[BoldFont={Songti SC Bold},ItalicFont={Songti SC Regular}]{Songti SC Regular}
\setCJKsansfont{Songti SC Light}
\setCJKfamilyfont{song}{Songti SC Regular}
\renewcommand{\CJKglue}{\hskip 0.05em}
\newcommand*{\songti}{\CJKfamily{zhsong}} % 宋体
\renewcommand{\refname}{\centerline{ \large \bf{参} \hspace{0.07cm} \bf{考}\hspace{0.07cm} \bf{文} \hspace{0.07cm}\bf{献}} }
%首行缩进
\setlength{\parindent}{2em}
\setmainfont{Times New Roman}
%行距
\renewcommand{\baselinestretch}{1.2} % 1.4倍行距
%页边距
\geometry{verbose,
  tmargin=2.27cm,% 上边距
  bmargin=2.27cm,% 下边距
  lmargin=2cm,% 左边距
  rmargin=2cm % 右边距
}

\definecolor{Blue}{RGB}{101,148,200}
%% 页眉页脚
\pagestyle{fancy}
\rhead{ \includegraphics[scale=0.19]{logo.jpg} } 
\chead{} % 页眉中间位置内容
\lhead{ \songti \textcolor{Blue}{\bf{GeWu Lab}}}
\renewcommand{\headrulewidth}{0.8pt}


%==========
% 正文部分
%==========

\begin{document}
~\\
\centerline{ \large \bf{每} \hspace{0.07cm} \bf{周}\hspace{0.07cm} \bf{研} \hspace{0.07cm}\bf{究} \hspace{0.07cm}\bf{展} \hspace{0.07cm}\bf{阶}\hspace{0.07cm} \bf{段}\hspace{0.07cm} \bf{汇} \hspace{0.07cm}\bf{报}}
~\\

%% \noindent 取消首行缩进
\noindent 汇报人:卫雅珂 \\
\noindent 电\hspace{0.29cm} 邮:weiyake@std.uestc.edu.cn \\
\noindent 时间段:2020年9月7日(周一) 至 2020年9月12日(周六) \\

\noindent 一、本周工作:

\begin{enumerate}[labelsep = .5em, leftmargin = -18pt, itemindent = 3em]
    \item [1.] 阅读参考文献\cite{tian2021unsupervised,NEURIPS2018_01161aaa,gao2018learning,uijlings2013selective,cinbis2016weakly}    
    \item [2.] 阅读《认知神经科学》第一章:认知神经科学简史;第二章:细胞机制与认知
    \item [3.] 阅读多模态综述剩余部分,及文中所提及的部分工作。
\end{enumerate}
\vspace{0.3cm}

\noindent 二、思考总结:

\noindent \textbf{Part 1.} 

\textbf{这是一个示例,文档支持中文和英文。中文为宋体,英文为Times New Roman。编译时:xe->bib->xe。中间可能会报错,但是不要紧,连续编译过后就可以生成正确的pdf文件。}

在《机器学习》第三章:线性模型中,多分类学习是之前没有接触过的内容。容易想到,将多分类任 务拆解为二分类任务就可以应用二分类方法,这也是解决多分类任务的一般方法。解决多分类任务关键是 如何对多分类任务进行拆分,以及如何对多个分类器进行集成。

\vspace{0.5cm}
\noindent \textbf{Part 2.} 

\textbf{这是一段引用示例,参考文献在refer.bib中。}
Training Deep Neural Networks is complicated by the fact \cite{diesase2012improving,levey2020nomenclature,hobson2015cost} that the distribution of each layer’s inputs changes during training, as the parameters of the previous layers change. This\cite{levey2020nomenclature} slows down the training by requiring lower learning rates and careful parameter initialization, and makes it notoriously hard to train models with saturating nonlinearities. We refer to this\cite{hobson2015cost} phenomenon as internal covariate shift, address the problem by normalizing layer in- puts. Our method draws its strength from making normalization a part of the model architecture and performing the normalization for each training mini-batch. Batch Normalization allows us to use much higher learning rates and be less careful about initialization. 

\vspace{0.5cm}
\noindent \textbf{Part 3.} 

\textbf{这是一段插入公式示例。}
We calculate AUC by Eq(\ref{equ:AUC}).

\begin{equation}
 \label{equ:AUC}
 AUC=\frac{\sum_{{ins_{i}\epsilon positiveclass}}rank_{ins_{i}}-\frac{M \times(M+1))}{2}}{M \times N}
\end{equation}

where $M$ is the number of positive class, and $N$ is the number of negative class. $rank_{ins_{i}}$ represents the possibility rank of sample $ins_{i}$ in the positive class. AUC indicates classifiers' ability to distinguish both positive and negative classes. Even in the condition of the highly imbalanced dataset, it can still put forward sensible evaluation.


\vspace{0.5cm}
\noindent 三、下周规划:
\begin{enumerate}[labelsep = .5em, leftmargin = -18pt, itemindent = 3em]
    \item [1.] 阅读周志华《机器学习》第三章:线性模型      
    \item [2.] 阅读《认知神经科学》第一章:认知神经科学简史;第二章:细胞机制与认知
    \item [3.] 阅读多模态综述剩余部分,及文中所提及的部分工作。
\end{enumerate}


\newpage

\bibliographystyle{IEEEtran}
\bibliography{refer}



\end{document}